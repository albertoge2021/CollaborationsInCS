\section{Conclusion}
\label{sec:conclusion}

In this study, we aimed to analyze the collaboration patterns and tendencies in the field of Computer Science for a long time, with special emphasis on China, the EU, and the US to provide a context on the status, benefits, and pitfalls for those collaborations.
First, we found that the US has been leading the research efforts in the field of Computer Science in the long term. However, it changed in recent years, when first China surpassed the US in 2011, only being exceeded by the US again in 2015, and second when the EU also surpassed the US in 2013. These results confirm the findings shared by the National Science Foundation placing China and the EU as the topmost productive regions in terms of number of publications in Science and Engineering \citep{burke2022state}. Because of this increase in Chinese publications, the number of co-authored articles between China and the US surpassed the previous leading collaborator partners, the EU and the US, in 2014. This change leaves China and the US as the top collaborators in Computer Science. Taking a deeper look into the different results obtained when analyzing the data based on the institution’s type that worked together, we observed that companies-only collaborations tend to be more internationalized, decreasing the number of self-collaborations only. This can be explained because the locations of headquarters or offices around the territories may help to increase the internationalization rate.
Second, we study the impact of the different regions when publishing their investigations, we found that those co-authored by the EU and the US tend to have the highest number of citations, suggesting that their findings tend to be more relevant than the others. In addition to this, results suggest that collaborating with the US also brings more citations for Chinese publications, obtaining twice as many as citations than the country obtains when publishing alone. These results, added that US-only papers obtained the second-highest average number of citations, indicate that the US has a big impact on their publications, either when the publication is alone or with institutions from other countries, which can be an important reason for the other countries to collaborate with the US. On the other hand, China and the EU benefit from collaboration between them too, both increasing their publication share when collaborating, compared to their works published when working within their countries only. Our results also indicate that articles published by Chinese companies were the least cited, suggesting that they might not be as relevant to other researchers as those published by colleagues from the EU or the US. In private sector collaborations, the US is clearly the most relevant, obtaining 39\% more citations on average than the second most cited, being those papers published by China and the EU.
Lastly, we found that China and the EU have been prioritizing “Artificial Intelligence”, “Physics”, “Mathematics”, and “Engineering” over others. This can also be seen when they have collaborated between themselves and the US. In contrast, the US has had not such a clear aim. Its research topics have been broader, including topics such as “Psychology” and “Medicine”. The EU has benefited from this wider approach by also collaborating with the US on these topics, bringing them more knowledge in the fields.
We can draw different conclusions for the three regions. First, while China produces the greatest number of scientific articles about Computer Science, their impact is relatively low. Hence, as observed in our study, China can benefit from collaborating with the EU and the US by obtaining more relevance in the scientific community and getting more quality in their publications. Its collaboration with the EU, whose interests seem to be aligned with China’s, can bring more, better, and more diversified studies for both partners. Apart from this, the EU can also take advantage of co-authorship with China by accessing the most productive region, in terms of the number of published papers. On the other hand, collaborating with the US can bring the EU high-quality knowledge about different topics, gaining more relevance in other fields. Lastly, the US can take advantage of collaborating with both China and the EU by accessing highly specialized research institutions, and in particular with the EU by creating high-impact publications, increasing the relevance of their institutions. Despite all these benefits for all the actors, politics can play a big role in collaborations between countries. Therefore, in future years, we will see which steps they decide to follow.